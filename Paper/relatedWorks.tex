\section{Related Works}
	Mining utility functions and discovering user preferences users has been explored in the literature before. Recommender systems, machine learning approaches in preference learning as well as preference elicitation are research fields that are closely related to our work.
	
	\textbf{Recommender systems}. Recommender systems \cite{Recommender:Rashid, Recommender:Burke} aim at predicting a new user's preferences based on the information about the previous users. Two general approaches is to either group the users together based on their similarities or to utilize the similarity in the contents of the items. Although approaches similar to recommender systems can be taken when inferring the distribution of the utility functions, the problem addressed in this project differs from the recommender systems in that it does not involve predicting a new user's preferences based on the information available about the new user. 
	
	\textbf{Preference learning}. Machine learning approaches \cite{GP:Chu, GP:Houlsby} have been developed to learn the and predict the preferences of a specific user. These approaches try to learn the utility function of a specific user based on the preferences expressed by the user and based on that, predict user's preferences on unseen items. These methods mainly focus on predicting user's preferences based on some previous knowledge. Although, in our approach, we can use models proposed in this area to learn a specific user's preferences, our focus is on understanding the distribution of the preferences of the users rather than predicting the preferences of the users on unseen data.

	\textbf{Preference elicitation}. Another area related to this topic is preference elicitation \cite{PE:Blum} in which queries are asked from the user to understand their preferences. This area, closer to preference learning, differs from our project in that we aim at understanding the user preferences from the ratings available and not asking questions from the users.

	The work in this project differs from the previously existing works in that, we analyze the provided ratings in the context of the items in the database. As such, our work is addressing a different problem compared to the vast majority of the literature that focuses on prediction of user preferences and recommendation of items to the customers based on the preferences.
